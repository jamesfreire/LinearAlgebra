\documentclass[12pt]{article}
\usepackage{amsmath}
\usepackage{amssymb}
\usepackage{geometry}
\usepackage{array}
\usepackage{fancyhdr}

\geometry{margin=1in}
\pagestyle{fancy}
\fancyhf{}
\rhead{Inverse of a 3×3 Matrix}
\lhead{Cofactor Method}
\cfoot{\thepage}

\title{\textbf{Inverse of a 3×3 Matrix - The Cofactor Method}}
\author{}
\date{}

\begin{document}

\maketitle

\section{Problem Statement}
Find the inverse of the following 3×3 matrix:
\[
A = \begin{pmatrix}
7 & -6 & 3 \\
4 & -5 & -4 \\
2 & 1 & 8
\end{pmatrix}
\]

\section{Step 1: Find the Determinant of Matrix A}

To find the determinant of a 3×3 matrix, we use the expansion along the first row with alternating signs (+, -, +):

\[
\det(A) = 7 \begin{vmatrix} -5 & -4 \\ 1 & 8 \end{vmatrix} - (-6) \begin{vmatrix} 4 & -4 \\ 2 & 8 \end{vmatrix} + 3 \begin{vmatrix} 4 & -5 \\ 2 & 1 \end{vmatrix}
\]

Calculate each 2×2 determinant:

\begin{align}
\begin{vmatrix} -5 & -4 \\ 1 & 8 \end{vmatrix} &= (-5)(8) - (-4)(1) = -40 + 4 = -36 \\
\begin{vmatrix} 4 & -4 \\ 2 & 8 \end{vmatrix} &= (4)(8) - (-4)(2) = 32 + 8 = 40 \\
\begin{vmatrix} 4 & -5 \\ 2 & 1 \end{vmatrix} &= (4)(1) - (-5)(2) = 4 + 10 = 14
\end{align}

Substituting back:
\[
\det(A) = 7(-36) + 6(40) + 3(14) = -252 + 240 + 42 = 30
\]

\section{Step 2: Find the Cofactors}

The cofactor $C_{ij}$ is given by $C_{ij} = (-1)^{i+j} M_{ij}$, where $M_{ij}$ is the minor (determinant of the 2×2 matrix obtained by removing row $i$ and column $j$).

The sign pattern for a 3×3 matrix is:
\[
\begin{pmatrix}
+ & - & + \\
- & + & - \\
+ & - & +
\end{pmatrix}
\]

\subsection{Visual Representation of 2×2 Matrices for Each Position}

To visualize the process, here are all the 2×2 matrices that correspond to each position in the original 3×3 matrix:

\[
\begin{pmatrix}
\begin{vmatrix} -5 & -4 \\ 1 & 8 \end{vmatrix} & \begin{vmatrix} 4 & -4 \\ 2 & 8 \end{vmatrix} & \begin{vmatrix} 4 & -5 \\ 2 & 1 \end{vmatrix} \\[1.5em]
\begin{vmatrix} -6 & 3 \\ 1 & 8 \end{vmatrix} & \begin{vmatrix} 7 & 3 \\ 2 & 8 \end{vmatrix} & \begin{vmatrix} 7 & -6 \\ 2 & 1 \end{vmatrix} \\[1.5em]
\begin{vmatrix} -6 & 3 \\ -5 & -4 \end{vmatrix} & \begin{vmatrix} 7 & 3 \\ 4 & -4 \end{vmatrix} & \begin{vmatrix} 7 & -6 \\ 4 & -5 \end{vmatrix}
\end{pmatrix}
\]

Each 2×2 matrix is formed by eliminating the row and column of the corresponding position from the original 3×3 matrix.

\subsection{Calculate all cofactors:}

\begin{align}
C_{11} &= (+1) \begin{vmatrix} -5 & -4 \\ 1 & 8 \end{vmatrix} = +(-36) = -36 \\
C_{12} &= (-1) \begin{vmatrix} 4 & -4 \\ 2 & 8 \end{vmatrix} = -(40) = -40 \\
C_{13} &= (+1) \begin{vmatrix} 4 & -5 \\ 2 & 1 \end{vmatrix} = +(14) = 14
\end{align}

\begin{align}
C_{21} &= (-1) \begin{vmatrix} -6 & 3 \\ 1 & 8 \end{vmatrix} = -((-6)(8) - (3)(1)) = -(-48 - 3) = 51 \\
C_{22} &= (+1) \begin{vmatrix} 7 & 3 \\ 2 & 8 \end{vmatrix} = +((7)(8) - (3)(2)) = +(56 - 6) = 50 \\
C_{23} &= (-1) \begin{vmatrix} 7 & -6 \\ 2 & 1 \end{vmatrix} = -((7)(1) - (-6)(2)) = -(7 + 12) = -19
\end{align}

\begin{align}
C_{31} &= (+1) \begin{vmatrix} -6 & 3 \\ -5 & -4 \end{vmatrix} = +((-6)(-4) - (3)(-5)) = +(24 + 15) = 39 \\
C_{32} &= (-1) \begin{vmatrix} 7 & 3 \\ 4 & -4 \end{vmatrix} = -((7)(-4) - (3)(4)) = -(-28 - 12) = 40 \\
C_{33} &= (+1) \begin{vmatrix} 7 & -6 \\ 4 & -5 \end{vmatrix} = +((7)(-5) - (-6)(4)) = +(-35 + 24) = -11
\end{align}

The cofactor matrix is:
\[
C = \begin{pmatrix}
-36 & -40 & 14 \\
51 & 50 & -19 \\
39 & 40 & -11
\end{pmatrix}
\]

\section{Step 3: Find the Adjugate Matrix}

The adjugate matrix is the transpose of the cofactor matrix:
\[
\text{adj}(A) = C^T = \begin{pmatrix}
-36 & 51 & 39 \\
-40 & 50 & 40 \\
14 & -19 & -11
\end{pmatrix}
\]

\section{Step 4: Calculate the Inverse}

The inverse of matrix $A$ is given by:
\[
A^{-1} = \frac{1}{\det(A)} \cdot \text{adj}(A)
\]

Substituting our values:
\[
A^{-1} = \frac{1}{30} \begin{pmatrix}
-36 & 51 & 39 \\
-40 & 50 & 40 \\
14 & -19 & -11
\end{pmatrix}
\]

\[
A^{-1} = \begin{pmatrix}
\frac{-36}{30} & \frac{51}{30} & \frac{39}{30} \\
\frac{-40}{30} & \frac{50}{30} & \frac{40}{30} \\
\frac{14}{30} & \frac{-19}{30} & \frac{-11}{30}
\end{pmatrix}
\]

\section{Step 5: Simplify the Fractions}

Reducing to lowest terms:
\[
A^{-1} = \begin{pmatrix}
-\frac{6}{5} & \frac{17}{10} & \frac{13}{10} \\
-\frac{4}{3} & \frac{5}{3} & \frac{4}{3} \\
\frac{7}{15} & -\frac{19}{30} & -\frac{11}{30}
\end{pmatrix}
\]

\section{Summary}

The complete process for finding the inverse of a 3×3 matrix using the cofactor method involves:

\begin{enumerate}
\item Calculate the determinant of the original matrix
\item Find all nine cofactors using the appropriate sign pattern
\item Transpose the cofactor matrix to get the adjugate matrix
\item Multiply the adjugate matrix by $\frac{1}{\det(A)}$
\item Simplify the resulting fractions
\end{enumerate}

\textbf{Final Answer:}
\[
A^{-1} = \begin{pmatrix}
-\frac{6}{5} & \frac{17}{10} & \frac{13}{10} \\
-\frac{4}{3} & \frac{5}{3} & \frac{4}{3} \\
\frac{7}{15} & -\frac{19}{30} & -\frac{11}{30}
\end{pmatrix}
\]

\end{document}
