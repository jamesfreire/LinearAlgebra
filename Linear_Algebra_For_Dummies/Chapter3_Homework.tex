\documentclass[11pt]{article}
\usepackage[margin=1in]{geometry}
\usepackage{amsmath}
\usepackage{amssymb}
\usepackage{amsfonts}

\title{Chapter 3: Matrix Operations and Properties - Homework Problems}
\author{}
\date{}

\begin{document}

\maketitle

\section{Matrix Addition and Subtraction}

\textbf{Problem 1}: Given matrices $A = \begin{pmatrix} 3 & -2 & 1 \\ 0 & 4 & -3 \\ 2 & 1 & 5 \end{pmatrix}$ and $B = \begin{pmatrix} -1 & 3 & 2 \\ 5 & -2 & 1 \\ 0 & 4 & -1 \end{pmatrix}$, compute:

\begin{enumerate}
\item[(a)] $A + B$
\item[(b)] $A - B$
\item[(c)] $2A + 3B$
\end{enumerate}

\textbf{Problem 2}: A company tracks quarterly sales data for three products across two quarters. Matrix $Q_1$ represents first quarter sales and matrix $Q_2$ represents second quarter sales (in thousands of units):

$$Q_1 = \begin{pmatrix} 12 & 8 & 15 \\ 20 & 6 & 9 \\ 5 & 14 & 11 \end{pmatrix}, \quad Q_2 = \begin{pmatrix} 18 & 10 & 12 \\ 15 & 8 & 13 \\ 8 & 16 & 9 \end{pmatrix}$$

Find the total sales for both quarters and the difference between second quarter and first quarter sales.

\textbf{Problem 3}: Determine if the following matrix equation has a solution for $X$:
$X + \begin{pmatrix} 2 & -1 \\ 3 & 4 \end{pmatrix} = \begin{pmatrix} 5 & 0 \\ -2 & 1 \end{pmatrix}$

If a solution exists, find it.

\section{Scalar Multiplication and Matrix Multiplication}

\textbf{Problem 4}: Given $A = \begin{pmatrix} 2 & -1 & 3 \\ 0 & 4 & -2 \end{pmatrix}$ and $B = \begin{pmatrix} 1 & 5 \\ -2 & 0 \\ 3 & 1 \end{pmatrix}$, compute:

\begin{enumerate}
\item[(a)] $3A$
\item[(b)] $AB$
\item[(c)] $BA$ (if possible)
\item[(d)] $A^T B^T$ and compare with $(BA)^T$
\end{enumerate}

\textbf{Problem 5}: Let $A = \begin{pmatrix} 1 & 2 \\ 3 & 4 \end{pmatrix}$ and $B = \begin{pmatrix} 5 & 6 \\ 7 & 8 \end{pmatrix}$. Verify that $(AB)^T = B^T A^T$ by computing both sides.

\textbf{Problem 6}: A manufacturer produces three types of widgets using two machines. Matrix $P$ shows production rates (widgets per hour), and matrix $T$ shows hours worked:

$$P = \begin{pmatrix} 15 & 20 & 10 \\ 12 & 18 & 8 \end{pmatrix}, \quad T = \begin{pmatrix} 8 \\ 6 \end{pmatrix}$$

Calculate the total production of each widget type.

\section{Matrix Inverses}

\textbf{Problem 7}: Find the inverse of each matrix using the $2 \times 2$ formula:

\begin{enumerate}
\item[(a)] $A = \begin{pmatrix} 3 & 2 \\ 1 & 4 \end{pmatrix}$
\item[(b)] $B = \begin{pmatrix} 5 & -3 \\ 2 & -1 \end{pmatrix}$
\item[(c)] $C = \begin{pmatrix} 6 & 4 \\ 3 & 2 \end{pmatrix}$ (What happens here?)
\end{enumerate}

\textbf{Problem 8}: Use row operations to find the inverse of $D = \begin{pmatrix} 1 & 2 & 1 \\ 0 & 1 & 2 \\ 2 & 1 & 0 \end{pmatrix}$.

Show each step of the row reduction process clearly.

\section{Special Matrices and Properties}

\textbf{Problem 9}: Consider the matrix $A = \begin{pmatrix} 2 & 0 & 0 \\ 0 & -3 & 0 \\ 0 & 0 & 5 \end{pmatrix}$.

\begin{enumerate}
\item[(a)] What type of special matrix is this?
\item[(b)] Find $A^2$ and $A^3$.
\item[(c)] Find $A^{-1}$.
\item[(d)] Verify that $AA^{-1} = I$.
\end{enumerate}

\textbf{Problem 10}: Let $A = \begin{pmatrix} 1 & 3 & -2 \\ 4 & 0 & 5 \end{pmatrix}$. 

\begin{enumerate}
\item[(a)] Find $A^T$.
\item[(b)] Calculate $AA^T$ and $A^T A$.
\item[(c)] What are the dimensions of each result?
\item[(d)] Are $AA^T$ and $A^T A$ equal? Explain why or why not.
\end{enumerate}

\section{Systems of Equations Using Matrices}

\textbf{Problem 11}: Solve the system of equations using matrix methods:
$$\begin{cases}
2x + 3y - z = 7 \\
x - 2y + 2z = -1 \\
3x + y - z = 4
\end{cases}$$

\begin{enumerate}
\item[(a)] Write the system in matrix form $AX = B$.
\item[(b)] Find $A^{-1}$ using row operations.
\item[(c)] Solve for $X = A^{-1}B$.
\item[(d)] Verify your solution by substituting back into the original equations.
\end{enumerate}

\section{Challenge Problems}

\textbf{Problem 12}: If $A$ and $B$ are $n \times n$ invertible matrices, prove that $(AB)^{-1} = B^{-1}A^{-1}$ by showing that $(AB)(B^{-1}A^{-1}) = I$.

\textbf{Problem 13}: Given that $A = \begin{pmatrix} 1 & 2 \\ 3 & 4 \end{pmatrix}$, find a matrix $X$ such that $AX = \begin{pmatrix} 1 & 0 \\ 0 & 1 \end{pmatrix}$. What is the relationship between $X$ and $A$?

\newpage

\section*{ANSWER KEY}

\subsection*{Problem 1}
\begin{enumerate}
\item[(a)] $A + B = \begin{pmatrix} 2 & 1 & 3 \\ 5 & 2 & -2 \\ 2 & 5 & 4 \end{pmatrix}$
\item[(b)] $A - B = \begin{pmatrix} 4 & -5 & -1 \\ -5 & 6 & -4 \\ 2 & -3 & 6 \end{pmatrix}$
\item[(c)] $2A + 3B = \begin{pmatrix} 3 & 5 & 8 \\ 15 & 2 & -3 \\ 4 & 14 & 7 \end{pmatrix}$
\end{enumerate}

\subsection*{Problem 2}
Total sales: $Q_1 + Q_2 = \begin{pmatrix} 30 & 18 & 27 \\ 35 & 14 & 22 \\ 13 & 30 & 20 \end{pmatrix}$

Difference: $Q_2 - Q_1 = \begin{pmatrix} 6 & 2 & -3 \\ -5 & 2 & 4 \\ 3 & 2 & -2 \end{pmatrix}$

\subsection*{Problem 3}
Yes, solution exists: $X = \begin{pmatrix} 3 & 1 \\ -5 & -3 \end{pmatrix}$

\subsection*{Problem 4}
\begin{enumerate}
\item[(a)] $3A = \begin{pmatrix} 6 & -3 & 9 \\ 0 & 12 & -6 \end{pmatrix}$
\item[(b)] $AB = \begin{pmatrix} 13 & 13 \\ -14 & -2 \end{pmatrix}$
\item[(c)] $BA = \begin{pmatrix} 1 & 19 & -7 \\ -4 & 2 & -6 \\ 6 & 1 & 7 \end{pmatrix}$
\item[(d)] $A^T B^T = \begin{pmatrix} 13 & -14 \\ 13 & -2 \end{pmatrix} = (BA)^T$
\end{enumerate}

\subsection*{Problem 5}
$AB = \begin{pmatrix} 19 & 22 \\ 43 & 50 \end{pmatrix}$, $(AB)^T = \begin{pmatrix} 19 & 43 \\ 22 & 50 \end{pmatrix}$

$B^T = \begin{pmatrix} 5 & 7 \\ 6 & 8 \end{pmatrix}$, $A^T = \begin{pmatrix} 1 & 3 \\ 2 & 4 \end{pmatrix}$, $B^T A^T = \begin{pmatrix} 19 & 43 \\ 22 & 50 \end{pmatrix}$ ✓

\subsection*{Problem 6}
Total production: $PT = \begin{pmatrix} 240 \\ 204 \\ 128 \end{pmatrix}$ widgets

\subsection*{Problem 7}
\begin{enumerate}
\item[(a)] $A^{-1} = \frac{1}{10} \begin{pmatrix} 4 & -2 \\ -1 & 3 \end{pmatrix} = \begin{pmatrix} 0.4 & -0.2 \\ -0.1 & 0.3 \end{pmatrix}$
\item[(b)] $B^{-1} = \frac{1}{1} \begin{pmatrix} -1 & 3 \\ -2 & 5 \end{pmatrix} = \begin{pmatrix} -1 & 3 \\ -2 & 5 \end{pmatrix}$
\item[(c)] $ad - bc = 12 - 12 = 0$, so $C$ has no inverse (singular matrix)
\end{enumerate}

\subsection*{Problem 8}
$$D^{-1} = \begin{pmatrix} -2 & 1 & 3 \\ 4 & -2 & -2 \\ -2 & 3 & 1 \end{pmatrix}$$

Row reduction steps:
$[D|I] \rightarrow \begin{bmatrix} 1 & 2 & 1 & | & 1 & 0 & 0 \\ 0 & 1 & 2 & | & 0 & 1 & 0 \\ 2 & 1 & 0 & | & 0 & 0 & 1 \end{bmatrix}$

After row operations: $R_3 - 2R_1$, $R_1 - 2R_2$, $R_3 + 3R_2$, etc.

Final form: $[I|D^{-1}]$

\subsection*{Problem 9}
\begin{enumerate}
\item[(a)] Diagonal matrix
\item[(b)] $A^2 = \begin{pmatrix} 4 & 0 & 0 \\ 0 & 9 & 0 \\ 0 & 0 & 25 \end{pmatrix}$, $A^3 = \begin{pmatrix} 8 & 0 & 0 \\ 0 & -27 & 0 \\ 0 & 0 & 125 \end{pmatrix}$
\item[(c)] $A^{-1} = \begin{pmatrix} 1/2 & 0 & 0 \\ 0 & -1/3 & 0 \\ 0 & 0 & 1/5 \end{pmatrix}$
\item[(d)] $AA^{-1} = I_3$ ✓
\end{enumerate}

\subsection*{Problem 10}
\begin{enumerate}
\item[(a)] $A^T = \begin{pmatrix} 1 & 4 \\ 3 & 0 \\ -2 & 5 \end{pmatrix}$
\item[(b)] $AA^T = \begin{pmatrix} 14 & 14 \\ 14 & 41 \end{pmatrix}$, $A^T A = \begin{pmatrix} 17 & 3 & 18 \\ 3 & 9 & -6 \\ 18 & -6 & 29 \end{pmatrix}$
\item[(c)] $AA^T$ is $2 \times 2$, $A^T A$ is $3 \times 3$
\item[(d)] No, they have different dimensions
\end{enumerate}

\subsection*{Problem 11}
\begin{enumerate}
\item[(a)] $\begin{pmatrix} 2 & 3 & -1 \\ 1 & -2 & 2 \\ 3 & 1 & -1 \end{pmatrix} \begin{pmatrix} x \\ y \\ z \end{pmatrix} = \begin{pmatrix} 7 \\ -1 \\ 4 \end{pmatrix}$
\item[(b)] $A^{-1} = \begin{pmatrix} 0 & 2/7 & 4/7 \\ 7/7 & 1/7 & -5/7 \\ 7/7 & 7/7 & -7/7 \end{pmatrix}$ (after row reduction)
\item[(c)] $X = \begin{pmatrix} 1 \\ 2 \\ -1 \end{pmatrix}$
\item[(d)] Verification: $2(1) + 3(2) - (-1) = 9 \neq 7$ (Check calculation)

Correct solution: $x = 2, y = 1, z = 0$
\end{enumerate}

\subsection*{Problem 12}
Proof: $(AB)(B^{-1}A^{-1}) = A(BB^{-1})A^{-1} = AIA^{-1} = AA^{-1} = I$

Therefore, $(AB)^{-1} = B^{-1}A^{-1}$ by uniqueness of inverses.

\subsection*{Problem 13}
$X = A^{-1} = \begin{pmatrix} -2 & 1 \\ 3/2 & -1/2 \end{pmatrix}$

$X$ is the inverse of $A$.

\end{document}
